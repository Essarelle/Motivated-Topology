\documentclass{report}

\usepackage[colorlinks=true, linkcolor=black, urlcolor=black]{hyperref}
\usepackage[utf8x]{inputenc}
\usepackage[inline]{asymptote}
\usepackage{graphicx}
\usepackage{amssymb}
\usepackage{amsmath}
\usepackage{multicol}
%\usepackage{amsthm}
\let\mathbbm=\mathbb
\newcommand{\cl}{\mathrm{cl}}
\newcommand{\intr}{\mathrm{int}}
\newcommand{\bd}{\mathrm{bd}}

\let\textDelta=\Delta

\setlength{\parindent}{0pt} % Default is 15pt.
\setlength{\parskip}{6pt plus1pt minus1pt}

%\newtheorem{proof}{Proof}

\begin{document}

\title{A Weird, Motivated, Intuitive, Introduction to Topology\\{\small(To Be Given a More Glamorous Title If It Amounts to Anything)}}
\author{Chris Olah\\ $~$\\ \href{mailto:colah@thielfellowship.org}{colah@thielfellowship.org} $~$\\ $~$\\ $~$\\ Creative Commons Attribution Share-Alike \\  \href{https://github.com/colah/Motivated-Topology}{github.com/colah/Motivated-Topology}\\ $~$\\}

\maketitle

\tableofcontents

\chapter*{Preface}

I call myself an aspiring mathematician because I'm painfully aware of how little I know compared to, for example, research mathematicians. I'm also aware that many introductions to topology have been written by people with much deeper understandings of the subject than me.

But after looking at many of the introductions available, I still feel discontent with what exists. Definitions feel unmotivated, point-set topology seems arbitrary. And at the same time, after spending spending many weeks mulling over the most basic ideas of topology, they've coalesced into very clear and obvious notions. I want to try and share that.

It occurs to me that perhaps I may have some comparative advantage in my lack of mathematical maturity, because I'm much closer to the level of someone first learning topology than is someone who has researched in it for decades. It occurs to me that perhaps I simply have a different aesthetic, caring deeply about motivation, and struggling with accepting definitions on my faith in the author that I will one day find the notion valuable. (Or perhaps the motivation is more easily seen by others, and so doesn't need to be taught. Or perhaps there's deep value in developing this intuition on ones own, and so it is strategically not taught. Or perhaps intuition is personal and difficult to convey.)

In the worst case, writing this is clarifying my own thoughts and understanding. {\bf As long as the reader understands this is very very rough} it seems unlikely to cause any harm.

And I think there's value in how radically different a perspective this is from the standard introduction to topology.

\chapter{Finding Closure}

\section{When Metrics Don't Matter}

When one thinks of distance, one thinks of distance between points on a plane. One thinks of distance between points on a sphere, like our Earth. Between points in space, and between points on a curve. But distance is so much more general than that! There is distance between functions, between sets, and between strings of text!\footnote{The names of these remarkable notions of distance, for the interested reader: For functions see L1, L2 and super norm based metrics. For sets, see Hausdorff distance. For strings, see the various versions of edit distance.}

Distance on the plane is part of our everyday life. When we walk around a room, move a pencil between points on a sheet of paper, or move a mouse on a computer, we are working with distance. Most of us have been drilled on questions relating to this from our earliest education -- the Pythagorean theorem is really about distance on a plane. If I move by $Δx$ horizontally and $Δy$ vertically, the distance that I moved from my starting point is the length of the diagonal line to my new position, which being the hypotenuse of the triangle we've made, is $\sqrt{Δx^2 + Δy^2}$. (Note that this wouldn't be true for triangles on a sphere.)

But things are different in different situations in life. A taxi driver in New York navigates a grid of roads and can't move diagonally. To the driver, the diagonal-movement-allowing notion of distance is very misleading: `closer' objects may be a longer drive away than `farther' objects. To the driver, distance between points is how far one needs to drive, the sum of the $x$ and $y$ distance, $|Δx| + |Δy|$.

As you can see, distance is a matter of context. There are often multiple equally valid notions of distance on the same space. We call these different notions of distance \emph{metrics}. The diagonal-movement-allowing metric is called the \emph{Euclidean metric}. The non-diagonal-movement-allowing metric is called the \emph{Manhattan metric}.

Once one has a metric, all sorts of questions become possible to ask. In addition to the obvious questions (is point $a$ or point $b$ further from $c$?), it becomes possible to formally define many ideas that we intuitively have:

\begin{itemize}
\item What is the boundary of a set?
\item Are two sets connected?
\item Is a function continuous?
\item Does a sequence converge?
\item Does a set have holes in it?
\end{itemize}

Remarkably, different though the Euclidean and the Manhattan metric are, they give the exact same answer to all these questions!

This observation hints that, for a certain class of broad and important questions, much of the information encoded in a notion of distance is superfluous and that some core similarity runs between all these different metrics. And, for me at least, such observations begin the thrill of the mathematical hunt. Can we precisely define this commonality between different metrics? Can we abstract away this core similarity and detach from notions of distance? If we succeed in abstracting this notion away, will it allow us to bring these notions to spaces that we can't put a useful metric on? The answer to all of these is yes!

If you think about it for a moment, you may begin to suspect that all this is entangled with what it means to be very very close under a certain metric. The Euclidean and Manhattan metric may disagree with specific issues, but they agree that the sequence of points $$(0.1,0.1), ~ (0.01,0.01), ~ (0.001,0.001) ~...$$ gets arbitrarily close to $(0,0)$. Since a lot of the issues we've talked about, like continuity, boundary, and connectedness involve infinitesimal behavior, ``arbitrary closeness'' seems like rather promising avenue.

On this intuition, we will pursue this idea of being arbitrarily or infinitely close. As it turns out, this is the field of topology. From this initial mission statement, it may seem like a very sparse or dry topic, but as we shall see it is actually exceedingly rich. In some ways, one might thing of topology as giving us a sort of ``qualitative geometry" where we escape concerns like specific length and angles. The ideas are also remarkable for their extreme generality

You may have previously heard things like: ``A topologist is someone who can't tell the difference between a coffee cup and a donut, because you can stretch and squish one into another.'' There's some truth to this. If you can imagine deforming one clay object into another, then they are topologically equivalent. This is because, while the distances between points may change, the local ``arbitrary closeness'' is unaffected and so when you forget about distance and only remember this arbitrary closeness, they are the same object. That said, topological equivalence doesn't actually require these imaginary deformations. And in fact, this image of topology really sells short the field.

At this point, you are hopefully persuaded that this ``arbitrary closeness'' is a worthwhile topic of investigation, even if you are somewhat skeptical of the claimed richness. As such, it is an appropriate time for us to nail down some more precise definitions. The correct definition is a complicated issue, often without a right or wrong answer,\footnote{See \emph{Proofs and Refutations} by Lakatos et al.} and we will evolve more sophisticated definitions of a topological space as we progress, but for now we shall base our inquiry on the notion of adherent points.

A point is called an \emph{adherent point} of a set if it is in or infinitely close to that set. That is, $p$ is an adherent point of $S$ if one can find $q ∈ S$ that is arbitrarily close to $p$. The \emph{closure} of a set $S$, $\cl(S)$, is all the adherent points of $S$. You can think of it as $S$, plus any parts of its boundary that it was missing.

A topological space $(X, \cl)$, then, is a space $X$ equipped with a closure operator $\cl$.

\section{The Abstract Axiomatic Approach}

One of the most famous mathematicians of all time is Euclid. He wrote a book, Euclid's \emph{Elements}, proving a large number of geometric results. Most of the results of Euclid's \emph{Elements} were proved before him by others, so why is it such a revered book? The true remarkableness of it lies not in the proved results, but in the way it begins with a few trivial geometric facts (eg. ``a line can be drawn between any two points'') that everyone can easily agree to be true, and builds up all the results as a logical progression from there.\footnote{This section is inspired by the section `Axioms and Men' from Pinter's \emph{A Book of Abstract Algebra}}

This approach of building results up from simple premises that are taken to be true, called \emph{axioms}. A consequence of this axiomatic approach is that it makes truth very objective, in some sense. Once you and I agree on the axioms, we can agree on whether or not a given result is true.
%\footnote{This sort of objective agreement on truth given prior agreement on certain basic things has some nice generalizations to probabilistic conditions and ``real life''. See Aumann's Agreement Theorem and friends.}
Further, even if you don't agree with me on the axioms, we should be able to agree on whether results are true \emph{were} the axioms true.

This axiomatic approach shouldn't be seen as constraining. It doesn't mean that every proof needs to be stated in terms of axioms, since we can use anything we've previously proven. And often informal discussion and intuition will not involve axioms at all. And it leaves you free to consider any system of axioms you want, and imagine what that world would be like!

But not only is it not constraining, building from axioms can be empowering. A great deal can be revealed about a topic by boiling it down to its core premises, on which everything is based. One sees how many of ideas really encode the same information. One can look at what changes as axioms are added and removed. It's really quite remarkable how a few simple sentences can contain a vast topic!

It's worth noting that, very frequently, axioms don't just float free, but are baked into the definition of something. If you define something to be an object with certain properties, those are `axioms' when you are talking about that type of object. In these cases, axioms become less a matter of truth and more a question of taste or usefulness.\footnote{Again, I refer the reader to \emph{Proofs and Refutations}, a play about mathematical philosophy where characters explore possible definitions of a polyhedron, and the process of coming up with those definitions.}

Alone, all this makes the axiomatic approach very compelling... But it actually develops into something much deeper. One recurring theme in mathematics is that we see the same patterns and ideas in very different places.

For example, you are familiar with the distributivity of multiplication: $a*(b+c) = a*b+a*c$. This is true not only about numbers, but also about matrices. It is also the case that for three sets, intersection distributes over union, $a∩(b∪c) = a∩b ∪ a∩c$, and that for three boolean values, and distributes over or, $a∧(b∨c) = a∧b ∨ a∧c$.

Another example is all the different mathematical objects we can talk about distance between.

These recurring patterns and ideas motivate mathematicians to abstract the common parts into a new idea. Not only are these generic ideas often beautiful and reveal very deep properties of objects... They also make our lives a lot easier, because we don't have to write the same proof out hundreds of times, for all the different objects we might want to work with that have the same properties!

(Readers with a background in programming or computer science may like to think of this as making mathematical proofs polymorphic, proving things not for a singular type of object, but for any type that extends a class, implements an interface, or the analogous idea in one's preferred programming language.\footnote{This analogy between proofs and programs is actually very deep. See the Curry-Howard Correspondence, which links proofs of theorems to programs generating objects of a particular type.})

%Spaces
These ``abstract structures'' are a staple of mathematics. Some taste of the variety that exists is valuable. There are spaces, sets equipped with some extra structure to give additional information about the space:

\begin{itemize}
\item Metric Spaces: a \emph{metric}, $d$, function gives distances between elements
\item Measure Spaces: a \emph{measure} function, $\mu$, gives the \emph{measure} -- a generalization of length/area/volume/... -- of subsets
\item Probability Spaces: measure spaces with a total measure of one
\item Topological Spaces: the topic of this book
\end{itemize}

There are also algebraic structures. These arose as mathematicians began to study algebraic questions involving things other than numbers, such as matrices and boolean values. During this explosion of algebras, it was observed that many of these different types of objects shared similar algebraic properties. These common properties were abstracted into algebraic structures. Perhaps the most notable such structures are sets along with binary operations that combine two elements into one in manners paralleling common arithmetic operations (like addition and multiplication):

\begin{itemize}
\item Monoids: an operation analogous to addition or multiplication
\item Groups: operations analogous to addition and subtraction (or multiplication and division)
\item Rings: operations analogous to addition, subtraction, and multiplication
\item Fields: operations analogous to addition, subtraction, multiplication and division
\item Vector Spaces: an `addition' operation and a `scaling' operation
\end{itemize}

Vector spaces often have additional structure:

\begin{itemize}
\item Normed Vector Spaces: vectors have a notion of length
\item Inner Product Spaces: a ``multiply parallel components'' operation gives elements lengths and angles between elements
\end{itemize}

These are only the most basic and foundation abstract structures in mathematics: hundreds more exist. Each one mentioned here is extremely deep and worthy of books of discussion -- any attempt to truley introduce them here would be a tragic butchering of the topic. They're mentioned to give you some taste for the variety of structures that exist. 

\subsection*{Exercises}

\begin{enumerate}

\item Pick an abstract structure you are unfamiliar with that was mentioned in this section. Imagine some reasonableness rules it might be subject to. Brainstorm a list of nice properties for it to have. Then look up its definition and see if you were right! Try to understand the motivation where the definition differs from yours. 

Is there another structure that is equivalent to what you came up with? You may want to look at Wikipedia's \href{http://en.wikipedia.org/wiki/Outline_of_algebraic_structures}{Outline of Algebraic Structures}. If, ultimately, you can't find your structure, you may have stumbled upon something new to explore!

(Repeat this exercise as desired, until you run out of unknown structures!)

\item Pick an abstract structure you are familiar with. Try to define it in the most radically different way possible. Extra points for more alternative definitions. Extra points for inventing new words in the process.

(Repeat this exercise as desired, until you run out of known structures!)

\end{enumerate}

\section{Axiomatization of Topological Spaces}

What sort of restrictions should be placed on a valid topological space?

Let's start by trying to ask concrete questions. For any space, there are two sets that are necessarily available for us to talk about: the empty set, $∅$, and the full space, $X$. What should the closure of these be?

To say that $\cl(∅)$ is anything other than the empty set doesn't really seem reasonable. It is equivalent to saying that some points in our space are on the boundary of \emph{nothing}. Therefore, 
\begin{equation}
\cl(∅) = ∅ \tag{Closure 1}
\end{equation}

To say that $\cl(X)$ is anything other than $X$ doesn't really seem reasonable either. The only other possibility would be for the closure to be a subset of $X$, which would be to say that points in $X$ aren't infinitely close to it, that adding the boundary removes point from the set. That would be pretty silly. And it brings us to a more general principle: the closure of $S$ should always include $S$,
\begin{equation}
S ⊆ \cl(S) \tag{Closure 2}
\end{equation}

Another natural question to ask, when given a function that takes objects of one type to the same type, is what happens if we apply it repeatedly. What happens if we iterate? What kind of function is $\cl ∘ \cl ∘ \cl ∘ ...$? Well, suppose $x ∈ \cl(\cl(S))$. Then $x$ is `infinitely close' to points in $\cl(S)$, which are `infinitely close' to points in $S$. It seems reasonable to think that the property of being `infinitely close' should be associative, and that $x$ should then be infinitely close to points in $S$. So, $x ∈ \cl(\cl(S))$ implies that $x ∈ \cl(S)$. Combined with the fact that $S ⊆ \cl(S)$, it must be the case that 
\begin{equation}
\cl(\cl(S)) = \cl(S) \tag{Closure 3}
\end{equation}
. That is, adding the boundary of a set to a set doesn't create a larger boundary. And in answer to our previous question, $\cl ∘ \cl ∘ \cl ∘ ... = \cl$.

Yet another natural question to ask is how closure interacts with our standard set operations. We will explore its relationship to complement shortly, but for now leave ourselves to consider only union and intersection.

Clearly, the closure of the union of $A$ and $B$ should contain the closure of $A$ and the closure of $B$ -- that is $\cl(A)∪\cl(B) ⊆ \cl(A∪B)$. But should the closure contain additional points? Can a point be `infinitely close' to the union of $A$ and $B$ without being `infinitely close' to $A$ or $B$? Surely it can only be as close to the union as it is to the closer of the two, and no closer than that? Again, for it to be another way doesn't seem very reasonable. So, 
\begin{equation}
\cl(A)∪\cl(B) = \cl(A∪B) \tag{Closure 4}
\end{equation}

What about intersection? A quick example will persuade you that there is little to be said here. The issue is that the two sets could be getting close to the set in very different ways. For example, consider the real numbers under and our usual notion of closure. The real numbers greater than zero, $ℝ_{>0}$, and real numbers less than zero, $ℝ_{<0}$, both get infinitely close to $0$, but they do so from opposite sides and so the intersection is empty and the closure of that is empty. Therefore, $\cl(A)∩\cl(B) ≠ \cl(A∩B)$.

Another natural property to desire is that $A ⊆ B$ should imply $\cl(A) ⊆ \cl(B)$. This, in fact, follows from our axioms: We can factor $B$ into $B = A∪(B\setminus A)$ and since $\cl(A∪(B\setminus A)) = \cl(A)∪\cl(B\setminus A)$, it follows that $\cl(B)$ must contain $\cl(A)$. This gives us our first lemma:
\footnote{Category Theorists: Notice that this makes closure an endofunctor on $(X, ⊆)$. In fact, closure is a monad.}
\begin{equation}
A ⊆ B   \implies   \cl(A) ⊆ \cl(B)   \tag{Lemma 1}
\end{equation}
  

These four properties we have come up with are called the \emph{Kuratowski closure axioms} or just the \emph{closure axioms}. They are (equivalent to) the definition of a topological space.

Now, you may feel there is a certain arbitrariness to selecting these axioms for our definition. Sure, they seem like reasonable axioms, but why them and not others? And you're absolutely correct. They are kind of arbitrary.

The thing is, mathematician's also talk about weaker versions of topological spaces (with few axioms) and stronger ones (with additional axioms). For example, a pre-topological space is a topological space without the requirement that $\cl(\cl(S)) = \cl(S)$. On the other hand a $T1$ topological space is one which has the additional requirement that $\cl(\{x\}) = \{x\}$, that no other points are `infinitely close' to an individual point. One is free to study any of these that they like. It just that topological spaces seem to be a healthy compromise to start at.

\section{Every Set Can Be A Topological Space!}

At this point, the only topological spaces you know of, really, are metric spaces you were already familiar with. In this section, we shall describe some topologies you can put on any space without making reference to metrics.

In a space $X$ with the indiscrete topology, the closure operator is $$\cl(S) = \begin{cases} ~∅ & S = ∅\\ ~X & S \neq ∅\\ \end{cases} $$ Intuitively, this means that every point is infinitely close to every other point. Notice that we haven't needed to talk about metrics to describe this and, in fact, can not as no metric corresponds to this topology, since metrics require there to be non-zero distance between unequal elements.\footnote{A metric without this pesky requirement that non-zero distance between unequal elements is called a \emph{pseudometric}. The pseudometric $d(x,y) = 0$ gives the indiscrete topology.} Now, we should rigorously verify that this indiscrete topology by checking that the proposed closure function obeys the closure axioms:

{\bf Proof}: We consider each of the four closure axioms independently.
\begin{enumerate}
\item By definition, $\cl(∅) = ∅$.
\item If $S=∅$ then $∅ ⊆ ∅$ and if $S \neq ∅$ then $S ⊆ X$, so $S ⊆ \cl(S)$.
\item If $S=∅$ then $\cl(\cl(∅)) =  \cl(∅)$ and if $S \neq ∅$ then $\cl(\cl(S)) =  \cl(X) = X = \cl(S)$, so $\cl(\cl(S)) = \cl(S)$.
\item If either $A$ or $B$ are $∅$, we may assume without loss of generality that $A$ is $∅$. Then $\cl(A∪B) = \cl(∅∪B) = \cl(B) = ∅ ∪ \cl(B) = \cl(A) ∪ \cl(B)$. 

On the other hand, if neither $A$ or $B$ is $∅$, then $\cl(A∪B) = X = X∪X = \cl(A) ∪ \cl(B)$. So, $\cl(A∪B) =  \cl(A) ∪ \cl(B)$.
\end{enumerate}

The opposite of the indiscrete topology is the discrete topology. Where in the indiscrete topology, everything is intuitively close together and sticks to everything else, in the discrete topology everything is spaced apart and touches nothing else. The closure operator for a discrete topology is the identity operation, $\cl(S) = S$, since there is never anything additional near by.

Another interesting topology is the finite-closed topology. It's sort of in between the discrete and indiscrete topologies. In it, nothing sticks to individual points, but all points are close to any infinite set. The closure operator is $$\cl(S) = \begin{cases} ~S & |S| ∈ ℕ\\ ~X & ~S~ \not\in ℕ\\ \end{cases} $$

\subsection*{Exercises}

\begin{enumerate}

\item Prove that the discrete topology's closure operation obeys the closure axioms.
\item Prove that the finite-closed topology's closure operation obeys the closure axioms.
\end{enumerate}

\section{Topological Operations: Interior, Boundary...}

Up to this point, we've only talked about closure. You may be getting tired of it. Never fear, we're about to greatly expand the topological operations we have available.

To begin with, let us define the \emph{boundary} of a set as the points that adhere to both the set and its complement. $$\bd(S) = \cl(S)∩\cl(S^C)$$

We will also define the \emph{interior} operator, which will be the opposite of our closure operator. Instead of adding the boundary to a set, it removes the boundary from a set. We can find the interior of a set by considering the closure of its complement. By adding the boundary to the complement of the set, when we complement again we get our set without the boundary. $$\intr(S) = \cl(S^C)^C$$

Finally, the limit points of a set are the parts of the closure that are not in the original set. $$\lim(S) = \cl(S) \setminus S$$

\subsection*{Topological Anatomy, A Visual Approach}

To build some intuition, let's consider an example: the space is a rectangle, with our set as the bottom half, its complement as the top. They both contain part of the small stripe boundary.

\begin{center}
\begin{asy}
size(0,3cm);

real sideY = 10;
real stripeY = 4;
real width = 30;
real mid = 15;

pair cBL = (0,0);
pair cTL = (0, 2*sideY + stripeY);
pair cBR = (width, 0);
pair cTR = (width, 2*sideY + stripeY);

pair mBL = (0, sideY);
pair mTL = (0, sideY + stripeY);
pair mBM = (mid, sideY);
pair mTM = (mid, sideY + stripeY);
pair mBR = (width, sideY);
pair mTR = (width, sideY + stripeY);

path c1=cBL--mBL--mBM--mTM--mTR--cBR--cycle;
path c2=cTL--mBL--mBM--mTM--mTR--cTR--cycle;
fill(c1,lightblue);
fill(c2,lightgreen);
draw(c1);
draw(c2);
draw(mBL--mBR, dashed);
draw(mTL--mTR, dashed);
label("$A$", (mid, sideY/2) );
label("$A^C$", (mid, stripeY+3*sideY/2) );

\end{asy}
%\caption{Venn diagram}\label{venn}
\end{center}

Let's now consider what happens to the set when we apply a number of operations.

\begin{center}
\begin{asy}
size(0,3cm);

real sideY = 10;
real stripeY = 4;
real width = 30;
real mid = 15;

pair cBL = (0,0);
pair cTL = (0, 2*sideY + stripeY);
pair cBR = (width, 0);
pair cTR = (width, 2*sideY + stripeY);

pair mBL = (0, sideY);
pair mTL = (0, sideY + stripeY);
pair mBM = (mid, sideY);
pair mTM = (mid, sideY + stripeY);
pair mBR = (width, sideY);
pair mTR = (width, sideY + stripeY);

picture p1 = new picture;
picture p2 = new picture;
picture p3 = new picture;
picture p4 = new picture;
picture p5 = new picture;
picture p6 = new picture;

path c1=cBL--mBL--mBM--mTM--mTR--cBR--cycle;
path c2=cTL--mBL--mBM--mTM--mTR--cTR--cycle;
fill(p1,c1,lightblue);
fill(p1,c2,lightgreen);
draw(p1,c1);
draw(p1,c2);
draw(p1,mBL--mBR, dashed);
draw(p1,mTL--mTR, dashed);
label(p1,"$A$", (mid, sideY/2) );
label(p1,"$A^C$", (mid, stripeY+3*sideY/2) );

path c1=cBL--mTL--mTR--cBR--cycle;
path c2=cTL--mTL--mTR--cTR--cycle;
fill(p2,c1,lightblue);
fill(p2,c2,lightgreen);
draw(p2,c1);
draw(p2,c2);
draw(p2,mBL--mBR, dashed);
draw(p2,mTL--mTR, dashed);
label(p2,"$\cl(A)$", (mid, sideY/2) );
label(p2,"$\intr(A^C)$", (mid, stripeY+3*sideY/2) );

path c1=cBL--mBL--mBR--cBR--cycle;
path c2=cTL--mBL--mBR--cTR--cycle;
fill(p3,c1,lightblue);
fill(p3,c2,lightgreen);
draw(p3,c1);
draw(p3,c2);
draw(p3,mBL--mBR, dashed);
draw(p3,mTL--mTR, dashed);
label(p3,"$\intr(A)$", (mid, sideY/2) );
label(p3,"$\cl(A^C)$", (mid, stripeY+3*sideY/2) );

add(p1);
add(shift((35,0))*p2);
add(shift((2*35,0))*p3);

\end{asy}

$~$\\

\begin{asy}
size(0,3cm);

real sideY = 10;
real stripeY = 4;
real width = 30;
real mid = 15;

pair cBL = (0,0);
pair cTL = (0, 2*sideY + stripeY);
pair cBR = (width, 0);
pair cTR = (width, 2*sideY + stripeY);

pair mBL = (0, sideY);
pair mTL = (0, sideY + stripeY);
pair mBM = (mid, sideY);
pair mTM = (mid, sideY + stripeY);
pair mBR = (width, sideY);
pair mTR = (width, sideY + stripeY);

picture p4 = new picture;
picture p5 = new picture;
picture p6 = new picture;

path c1=cBL--mTL--mTR--cBR--cycle;
path c2=cTL--mBL--mBM--mTM--mTR--cTR--cycle;
path c3=mTL--mBL--mBM--mTM--cycle;
fill(p4,c1,lightblue);
fill(p4,c2,lightgreen);
fill(p4,c3,lightred);
draw(p4,c1);
draw(p4,c2);
draw(p4,mBL--mBR, dashed);
draw(p4,mTL--mTR, dashed);
label(p4,"$A$", (mid, sideY/2) );
label(p4,"$\intr(A^C)$", (mid, stripeY+3*sideY/2) );
label(p4,"$\lim(A)$", (mid/2, stripeY/2+1.0*sideY) );

path c1=cBL--mTL--mTR--cBR--cycle;
path c2=cTL--mBL--mBR--cTR--cycle;
path c3=mTL--mBL--mBR--mTR--cycle;
fill(p5,c1,lightblue);
fill(p5,c2,lightgreen);
fill(p5,c3,lightred);
draw(p5,c1);
draw(p5,c2);
draw(p5,mBL--mBR, dashed);
draw(p5,mTL--mTR, dashed);
label(p5,"$\intr(A)$", (mid, sideY/2) );
label(p5,"$\intr(A^C)$", (mid, stripeY+3*sideY/2) );
label(p5,"$\bd(A)=\bd(A^C)$", (mid, stripeY/2+1.06*sideY) );

add(p4);
add(shift((35,0))*p5);
//add(shift((2*35,0))*p3);

\end{asy}
\end{center}

Now, this paints a bit of a falsely simplistic picture... ((Give example where things are less simple.))

\subsection*{Topological Operation Identities}

These pictures should have given you a lot of intuition for how these operations should interact. There are a lot of nice properties. We will call them the \emph{topological operation identities}.

As you look at these, look back at the pictures and see how these identities can be observed in them.

\begin{multicols}{2}

\subsubsection*{Closure -- Interior}

$$\cl(S)^C = \intr(S^C)$$
$$\cl(S^C) = \intr(S)^C$$

\subsubsection*{Closure -- Boundary}

$$\cl(S) = S∪\bd(S)$$

\subsubsection*{Closure -- Limit}

$$\cl(S) = S∪\lim(S)$$

\subsubsection*{Interior -- Boundary}

$$\intr(S) = S \setminus \bd(S)$$

$$\bd(S) = (\intr(S)∪\intr(S^C))^C$$

\subsubsection*{Interior -- Limit}

$$\intr(S) = S \setminus \lim(S^C) $$

\subsubsection*{Boundary -- Limit}

$$\bd(S) = \lim(S)∪\lim(S^C)$$

\end{multicols}

{\bf Proof (Closure -- Interior)}: The definition of interior gives us $\cl(S^C)^C = \intr(S)$. By substituting $S=S^C$ we get $\cl(S)^C = \intr(S^C)$. On the other hand, by complementing both sides, we get $\cl(S^C) = \intr(S)^C$.

The proof of the rest of the identities are left as an exercise to the reader.


\subsection*{Exercises}

\begin{enumerate}

\item Prove the remaining topological operation identities.

\item How many different sets can you make using closure and complement?

\item A topological space can be defined in terms of an interior operator as well as closure. What are the axioms for an interior-based definition? (eg. $\intr(\intr(S)) = \intr(S)$ )

\item A topological space can also be defined in terms of a boundary operator as well as closure. What are the axioms for a boundary-based definition? (Hint: one of the axioms becomes unnecessary.)

\end{enumerate}

\section{Topological Fixed-Points: Open and Closed}

An open set is one which is its own interior ($\intr(S) = S$). This means that it does not contain its boundary. Conversely, a closed set is one which is its own closure ($\cl(S) = S$). Closed sets completely contain their boundary.\footnote{The more general notion here is fixed-points. Closed sets are the fixed-points of closure and open sets are the fixed-points of interior. Fixed-points can be a very interesting topic.}

A clopen set is a set that is both open and closed. How can this be? How can a set both contain and not contain its boundary? The answer is that clopen sets \emph{don't have a boundary}. For example, the empty set $∅$ and the full space $X$ are both clopen sets, since they have no boundary. The existence of other clopen sets is tightly tied to notions of connectedness, since if some portion of a space does not have a boundary, it is not connected to the rest of the space.

It is important to note that not all sets are open or closed (or both). Many contain part of their boundary but not another part.

A natural question to ask is how closed and open sets interact with our basic set operations.

First we consider complement. If a set is closed, it's complement is open and vice versa. The idea is that if a set contains its boundary, it's complement doesn't contain the boundary.

{\bf Proof}: If $S$ is closed, then by definition $\cl(S) = S$. So $\intr(S^C) = \cl((S^C)^C)^C = \cl(S)^C = S^C$, which means $S^C$ is open.

Next, we consider union. The union of two closed sets is closed. 

{\bf Proof}: If $A$ and $B$ are closed, $\cl(A) = A$ and $\cl(B) = B$. Recall that one of our closure axioms is that $\cl(A∪B) = \cl(A)∪\cl(B)$. Then, $\cl(A∪B) = A∪B$ and $A∪B$ is closed.

Finally we consider intersection. The intersection of any number of closed sets is closed. The idea is that if a point is close to the intersection of the sets, it was close to all the sets, and since they were closed it was also an element of all those sets, and so in the intersection as well.

{\bf Proof}: If $\{S_n\}$ is a collection of closed sets, then $\cl(S_n) = S_n$ for all $n$. Let $S$ be the intersection of this collection, $S=\bigcap_n S_n$. Since $S ⊆ S_n$ for all $n$, by an earlier lemma, $\cl(S) ⊆ \cl(S_n)$ and since $\cl(S_n) = S_n$, it is the case that $\cl(S) ⊆ S_n$. It follows that $\cl(S) ⊆ \bigcap_n \cl(S_n) = S$ and $S = \cl(S)$. Therefore, $S$ is closed.

You've probably noticed that we proved that the union of \emph{two} closed sets is closed but that we proved that the intersection of an \emph{arbitrary number} of closed sets is closed? What's the difference? Doesn't an arbitrary number follow from our two result by mathematical induction?

Well, an arbitrary finite number does. For example, since the union of two closed sets is closed, the union of ten closed sets is also closed: union the first two, then than with the next one, and so on. But a union of an infinite number of closed sets is not necessarily closed. A simple example is that, in the real numbers, any set consisting of a single point is closed but it does not follow that every subset of the real numbers, being a union of singleton sets for every element, are all closed.

In conclusion:

\begin{itemize}
\item The complement of a closed set is an open set.
\item The union of two closed sets is a closed set.
\item The intersection of an arbitrary number of a closed sets is a closed set.
\end{itemize}

What about open sets? Because they are the complements of closed sets, each one of these results has a kind of `dual' result. Note how union and intersection switch. This is because of De Morgan's law: $(a ∪ b)^C = a^C ∩ b^C$ and $(a ∩ b)^c = a^C ∪ b^C$.

\begin{itemize}
\item The complement of an open set is a closed set.
\item The intersection of two open sets is an open set.
\item The union of an arbitrary number of open sets is an open set.
\end{itemize}

Proving these is left as an exercise to the reader.

\subsection*{Exercises}

\begin{enumerate}

\item Are the following subsets of the real numbers (with the usual topology) open, closed, clopen or neither?
\begin{enumerate}
\item $[0,1] = \{x | 0 ≤ x ≤ 1\}$
\item $[0,1) = \{x | 0 ≤ x < 1\}$
\item $(0,1) = \{x | 0 < x < 1\}$.
\end{enumerate}

\item Describe which sets are open and which sets are closed in:

\begin{enumerate}
\item The discrete topology.
\item The indiscrete topology.
\item The finite-closed topology.
\end{enumerate}

\item Prove the open set properties by application of De Morgan's law.

\item Prove the open set properties from scratch, using the \emph{interior axioms} you developed previously.

\item We've considered sets which are their own interior (open sets) and sets which are their own closure (closed sets), but what about sets that are their own boundary (boundary sets)? 
\begin{enumerate}
\item Are they open, closed, clopen, or neither (and how does this vary between examples)?
\item How do they interact with our basic set operations: complement, union, and intersection?
\end{enumerate}

\end{enumerate}

\section{What Does Closure Look Like?}

What does it \emph{really} mean to be a valid closure operator? We have a bunch of rules, but what do they mean?

One thing that we can do to really help us understand is look visualize what all the rules mean.

We will represent $A \subset B$ by a blue line between $A$ and $B$, with $A$ bellow $B$. We will represent $\cl(A) = B$ with a red arrow from $A$ to $B$. 

For example, recall that the discrete closure operator is the one where no points are close together, and so closure doesn't change anything. We visualize it on the space $\{1,2\}$ as:

\begin{center}
\includegraphics{images/ClosureAxiomsOnLattice/discrete.pdf}
\end{center}

On the other hand, in the indiscrete closure, where all points are close together and so every non-empty set gets mapped to the full space, looks like:

\begin{center}
\includegraphics{images/ClosureAxiomsOnLattice/indiscrete.pdf}
\end{center}

As we will see, all the closure axioms have visual interpretations. As you read, verify that the two previous examples obey the closure axioms.

\subsection*{Closure 1: $\cl(∅) = ∅$}

The empty set is mapped by closure to itself.

\begin{center}
\includegraphics{images/ClosureAxiomsOnLattice/empty.pdf}
\end{center}

\subsection*{Closure 2: $S ⊆ \cl(S)$}

Arrows either go horizontally (to itself) or upwards (to one of its super sets), but never downwards.

\begin{center}
\includegraphics{images/ClosureAxiomsOnLattice/superseteq.pdf}
\end{center}

\subsection*{Closure 3: $\cl(\cl(S)) = \cl(S)$}

If an arrow maps to some value, that value must map to itself. For example, if the closure of $A$ is $B$, then $B$ must be its own closure.

\begin{center}
\includegraphics{images/ClosureAxiomsOnLattice/idempotent.pdf}
\end{center}
 
\subsection*{Closure 4: $\cl(A)∪\cl(B) = \cl(A∪B)$}

The union of two sets maps to the union of their closures. For example, if $C$ is the closure of $A$ and $D$ is the closure of $B$, then $C \cup D$ must be the closure of $A \cup B$.

\begin{center}
\includegraphics{images/ClosureAxiomsOnLattice/union.pdf}
\end{center}

\section*{Insights from Visualization}

One very important insight to draw is just how far from being an arbitrary mapping from subsets of your space ($P(X)$) to subsets of your space the closure operation is. Only a tiny fraction of such functions obey the closure axiom.

For a set of cardinality (size) $n$, the number of possible maps from its subsets to its subsets is $(2^n)^{(2^n)}$. In the next chapter, we will see that the number of valid closure operators is much less than $2^{(2^n)}$. That's a pretty big difference!

This conflict between the number of possible mappings and the number of valid closure operators is a first hint that perhaps the closure operator isn't the best way to represent topological information about a set. 

\subsection*{Exercises}

\begin{enumerate}

\item Draw the discrete closure operator on $\{1,2,3\}$. Verify that it obeys the closure axioms visually.
\item Draw the inidiscrete closure operator on $\{1,2,3\}$. Verify that it obeys the closure axioms visually.
\item Draw a closure operator on $\{1,2,3\}$ that isn't the discrete or indiscrete closure operator.
\item Draw the finite-closed closure operator on $ℤ$. Verify that it obeys the closure axioms visually. (Obviously, since $ℤ$ is infinite, you can't draw the entire thing. Draw a represantitive part and imagine what the entire thing would be like...)
\item On the closure operator drawings you've made, circle or highlight the closed sets. What do you notice?
\item Prove that closed sets uniquely define the closure operation.
\item Visually prove the basic set interactions with closed sets. (Prove that the union of two closed sets is closed, that the intersection of an arbitrary number of closed sets is closed.)
\item The interior operator can be visualized in the same way. Redraw the previous examples with arrows representing interior (preferably, use a different color). Compare the closure diagrams and the interior diagrams. What do you notice?
\item On the interior operator drawings you've made, circle or highlight the open sets. What do you notice?
\item Prove that the open sets uniquely define the interior operator.



\end{enumerate}

\chapter{Finding Openess}

\section{Arbitrarily Close: Your Local Neighborhood Filter}

((Explain how open sets can be a drop in for ``within $\epsilon$ distance" and the neighborhood filter can replace ``for all $\epsilon > 0$ distance...".))

\section{Topology in Terms of Open Sets}

((Redefine closure using neighborhood filters. Prove that the open set properties we derived in the last chapter imply the closure axioms. Therefore, we can describe a topological space in terms of open sets.))

((Reasons why it might be better: More random structures form topologies; Easier to describe topologies; Useful in proofs; Open sets are a very powerful language for us to state more complicated ideas in. Remind readers that there can be lots of valid definitions.))

\section{Separation Axioms}

((Introduce separation axioms with pictures as in \href{http://christopherolah.wordpress.com/2010/08/14/separation-axiom-visualisations}{my blog post on separation axioms} .))

\chapter{Finding the Big Picture}

((Local point set topology isn't very glamorous. To prevent reader fatigue, we want to give some tools to zoom out and see the big picture of a topological space (AKA, give them something closer the popular notion of topology). So, we introduce some combinatorial or algebraic topology. Probably homotopy groups.))

\end{document}
