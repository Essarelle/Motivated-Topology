\documentclass{report}

\usepackage[utf8x]{inputenc}
\usepackage{amssymb}
\let\mathbbm=\mathbb
\newcommand{\cl}{\mathrm{cl}}
\newcommand{\intr}{\mathrm{int}}
\newcommand{\bd}{\mathrm{bd}}

\setlength{\parindent}{0pt} % Default is 15pt.
\setlength{\parskip}{6pt plus1pt minus1pt}

\begin{document}

\title{A Weird, Intuitive, Introduction to Topology\\{\small(To Be Given a More Glamorous Title If It Amounts to Anything)}}
\author{Chris Olah}

\maketitle

\chapter{Finding Closure}

\section{When Metrics Don't Matter}

When one thinks of distance, one thinks of distance between points on a plane. One thinks of distance between points on a sphere, like our Earth. Between points in space, and between points on a curve. But distance is so much more general than that! There is distance between fuctions, between sets, and between strings of text!\footnote{The names of these remarkable notions of distance, for the interested reader: For functions see L1, L2 and super norm based metrics. For sets, see Hausdorff distance. For strings, see the various versions of edit distance.}

And so, while at first glance distance may seem like a simple straight-forward idea it is actually a topic of great depth.

Before we go on, a little more conreteness. We will call things like these, which have a notion of distance, metric spaces. Formally, they are a set $X$ along with a function $d$ (the \emph{metric}) which assigns distances to pairs of points in $X$. $d$ must be reasonable in some sense that we will discuss later.

For example, the natural notion of distance on the real numbers corresponds to $d(a,b) = |a-b|$, the absolute value of their difference. So, $d(3,4) = |3-4| = 1$, $d(-1, -5) = |-1-(-5)| = 4$ and $d(2,2) = |2-2| = 0$.

Similarily, we can define the metric for our regular notion of distance on a plane, the Euclidean Metric, as $d(a,b) = \sqrt{(a_1-b_1)^2 + (a_2-b_2)^2}$. This follows from the Pythagorian Theorem. So, $d((0,0), (3,4)) = \sqrt{3^2+4^2} = 5$.

As the caveated language has suggested, however, this is not the only way to define distance. Imagine that you are driving a car in downtown Manhattan. The grid system of the roads means that you can only move horizontally and vertically, but not diagonally. For your purposes, something is as far away from you as the sum of its horizontal and vertical distnaces. This is the Manhattan Metric: $d(a,b) = |a_1-b_1| + |a_2-b_2|$. So, $d((1,2), (4, -1)) = |4-1| + |-1 -2| = 6$.

Different though these metrics seem, they actually agree on surprisingly great deal. The agree on what the boundary of sets are and whether sets are connected. They agree on which functions involving our space as its domain or range are continuous. The agree on whether our sets have holes in them.

These observations hint that, for a certain class of questions, much of the information encoded in a notion of distance is superfluous and that some core similarity runs between all these different metrics. And, for me at least, such observations begin the thrill of the mathematical hunt. Can we precisely define this commonality between different metrics? Can we abstract away this core similarity and detach from notions of distance? If we succeed in abstracting this notion away, will it allow us to bring these notions to spaces that we can't put a useful metric on? The answer to all of these is yes!

If you think about it for a moment, you may begin to suspect that all this is entangled with what it means to be very very close under a certain metric. The Euclidean and Manhatan metric may disagree with specific issues, but they agree that the sequence of points $$(0.1,0.1), ~ (0.01,0.01), ~ (0.001,0.001) ~...$$ gets arbitrarily close to $(0,0)$. Since a lot of the issues we've talked about, like continuity, boundary, and connectedness involve infintessimal behavior, ``arbitrary closeness'' seems like rather promising avenue.

On this intuition, we will pursue this idea of being arbitrarily or infinitely close. As it turns out, this is the field of topology. From this initial mission statement, it may seem like a very sparse or dry topic, but as we shall see it is actually exceedingly rich. In some ways, one might thing of topology as giving us a sort of ``qualitative geometry" where we escape concerns like specific length and angles. The ideas are also remarkable for their extreme generality

You may have previously heard things like: ``A topologist is someone who can't tell the difference between a coffee cup and a donut, because you can stretch and squish one into another.'' There's some truth to this. If you can imagine deforming one clay object into another, then they are topologically equivalent. This is because, while the distances between points may change, the local ``arbitrary closeness'' is unaffected and so when you forget about distance and only remember this arbitrary closeness, they are the same object. That said, topological equivalence doesn't actually require these imaginary deformations. And in fact, this image of topology really sells short the field.

At this point, you are hopefully persuaded that this ``arbitrary closeness'' is a worthwhile topic of investigation, even if you are somewhat skeptical of the claimed richness. As such, it is an appropriate time for us to nail down some more precise definitions. The correct definition is a complicated issue, often without a right or wrong answer,\footnote{See \emph{Proofs and Refutations} by Lakatos et al.} and we will evolve more sophisticated definitions of a topological space as we progress, but for now we shall base our inquiry on the notion of adherant points.

A point is called an \emph{adherant point} of a set if it is in or infinitely close to that set. That is, $p$ is an adherant point of $S$ if one can find $q ∈ S$ that is arbitrarily close to $p$. The \emph{closure} of a set $S$, $\cl(S)$, is all the adherant points of $S$. You can think of it as $S$, plus any parts of its boundary that it was missing.

A topological space $(X, \cl)$, then, is a space $X$ equipped with a closure operator $\cl$.

\section{The Abstract Axiomatic Approach}

One of the most famous mathematicians of all time is Euclid. He wrote a book, Euclid's \emph{Elements}, proving a large number of geometric results. Most of the results of Euclid's \emph{Elements} were proved before him by others, so why is it such a revered book? The true remarkableness of it lies not in the proved results, but in the way it begins with a few trivial geometric facts (eg. ``a line can be drawn between any two points'') that everyone can easily agree to be true, and builds up all the results as a logical progression from there.\footnote{This section is inspired by the section `Axioms and Men' from Pinter's \emph{A Book of Abstract One}}

Algebra consequence of axiomatic approach is that it makes truth very objective, in some sense. Once you and I agree on the axioms, we can agree on whether or not a given result is true.\footnote{This sort of objective agreement on truth given prior agreement on certain basic things has some nice generalizations to probabilistic conditions and ``real life''. See Aumann's Agreement Theorem and friends.} Further, even if you don't agree with me on the axioms, we should be able to agree on whether results are true \emph{were} the axioms true.

This axiomatic approach shouldn't be seen as constraining. It doesn't mean that every proof needs to be stated in terms of axioms, since we can use anything we've previously proven. And often informal discussion and intuition will not involve axioms at all. And it leaves you free to consider any system of axioms you want, and imagine what that world would be like!

But not only is the axiomatic approach not constraining, it can be empowering. A great deal can be revealed about a topic by boiling it down to its core premises, on which everything is based. One sees how many of ideas really encode the same information. One can look at what changes as axioms are added and removed.

It's worth noting that, very frequently, axioms don't just float free, but are baked into the definition of something. ...

Let's set aside axioms for a moment, and conduct a thought experiment. You are a mathematician a few centuries ago, interested in proving 

...

%Spaces
These ``abstract structures'' are a staple of mathematics. Some taste of the variety that exists is valuable. There are spaces, sets equipped with some extra structure to give additional information about the space:

\begin{itemize}
\item Metric Spaces: a \emph{metric} function gives distances between elements
\item Measure Spaces: a \emph{measure} function gives the \emph{measure} -- a generalization of length/area/volume/... -- of subsets
\item Probability Spaces: measure spaces with a total measure of one
\item Topological Spaces: the topic of this book
\end{itemize}

There are also algebraic structures. These arose as mathematicians began to study algebraic questions involving things other than numbers, such as matrices and boolean values. During this explosion of algebras, it was observed that many of these different types of objects shared similar algebraic properties. These common properties were abstracted into algebraic structures. Perhaps the most notable such structures are sets along with binary operations that combine two elements into one in manners paralleling common arithmetic operations (like addition and multiplication):

\begin{itemize}
\item Monoids: an operation analagous to addition or multiplication
\item Groups: operations analagous to addition and subtraction (or multiplication and division)
\item Rings: operations analagous to addition, subtraction, and multiplication
\item Fields: operations analagous to addition, subtraction, multiplication and divison
\item Vector Spaces: an `addition' operation and a `scaling' operation
\end{itemize}

Vector spaces often have additional structure:

\begin{itemize}
\item Normed Vector Spaces: vectors have a notion of length
\item Inner Product Spaces: a ``multiply parallel components'' operation gives elements lengths and angles between elements
\end{itemize}

These are only the most basic and foundation abstract structures in mathematics: hundreds more exist. Each one mentioned here is extrmely dep and worthy of books of discussion -- any attempt to truely introduce them here would be a tragic butchering of the topic. They're mentioned to give you some taste for the variety of structures that exist. 


... 

(Readers with a background in programming or computer science may like to think of this as making mathematical proofs polymorphic, proving things not for a singular type of object, but for any type that extends a class, implements an interface, or the analogous idea in one's preferred programming language.\footnote{This analogy between proofs and programs is actually very deep. See the Curry-Howard Correspondence, which links proofs of theorems to programs generating objects of a particular type.})

{\bf Exercise}: Pick an abstract structure you are unfamiliar with that was mentioned in this section. Imagine some reasonableness rules it might be subject to. Brainstorm a list of nice properties for it to have. Then look up its definition and see if you were right! Try to understand the motivation where the definition differs from yours. (Repeat this exercise as desired, until you run out of unknown structures!)

{\bf Exercise}: Pick an abstract structure you are familiar with. Try to define it in the most radically different way possible. Extra points for more alternative definitions. Extra points for inventing new words in the process.

\section{Axiomatization of Topological Spaces}

What sort of restrictions should be placed on a valid topological space?

Let's start by trying to ask concrete questions. For any space, there are two sets that are necessarily available for us to talk about: the empty set, $∅$, and the full space, $X$. What should the closure of these be?

To say that $\cl(∅)$ is anything other than the empty set doesn't really seem reasonable. It is equivalent to saying that some points in our space are on the boundary of \emph{nothing}. Therefore, $$\cl(∅) = ∅$$

To say that $\cl(X)$ is anything other than $X$ doesn't really seem reasonable either. The only other possibility would be for the closure to be a subset of $X$, which would be to say that points in $X$ aren't infinitely close to it, that adding the boundary removes point from the set. That would be pretty silly. And it brings us to a more general principle: the closure of $S$ should always include $S$, $$S ⊆ \cl(S)$$

Another natural question to ask, when given a function that takes objects of one type to the same type, is what happens if we apply it repeatedly. What happens if we iterate? What kind of function is $\cl ∘ \cl ∘ \cl ∘ ...$? Well, suppose $x ∈ \cl(\cl(S))$. Then $x$ is `infinitely close' to points in $\cl(S)$, which are `infinitely close' to points in $S$. It seems reasonable to think that the property of being `infinitely close' should be associative, and that $x$ should then be infinitely close to points in $S$. So, $x ∈ \cl(\cl(S))$ implies that $x ∈ \cl(S)$. Combined with the fact that $S ⊆ \cl(S)$, it must be the case that $$\cl(\cl(S)) = \cl(S)$$. That is, adding the boundary of a set to a set doesn't create a larger boundary. And in answer to our previous question, $\cl ∘ \cl ∘ \cl ∘ ... = \cl$.

Yet another natural question to ask is how closure interacts with our standard set operations. We will explore its relationship to complement shortly, but for now leave ourselves to consider only union and intersection.

Clearly, the closure of the union of $A$ and $B$ should contain the closure of $A$ and the closure of $B$ -- that is $\cl(A)∪\cl(B) ⊆ \cl(A∪B)$. But should the closure contain additional points? Can a point be `infinitely close' to the union of $A$ and $B$ without being `infinitely close' to $A$ or $B$? Surely it can only be as close to the union as it is to the closer of the two, and no closer than that? Again, for it to be another way doesn't seem very reasonable. So, $$\cl(A)∪\cl(B) = \cl(A∪B)$$

What about intersection? A quick example will persuade you that there is little to be said here. The issue is that the two sets could be getting close to the set in very different ways. For example, consider the real numbers under and our usual notion of closure. The real numbers greater than zero, $ℝ_{>0}$, and real numbers less than zero, $ℝ_{<0}$, both get infinitely close to $0$, but they do so from opposite sides and so the intersection is empty and the closure of that is empty. Therefore, $\cl(A)∩\cl(B) ≠ \cl(A∩B)$.

These four properties we have come up with are called the \emph{Kuratowski closure axioms} or just the \emph{closure axioms}. They are (equivalent to) the definition of a topological space.

Now, you may feel there is a certain arbitrariness to selecting these axioms for our definition. Sure, they seem like reasonable axioms, but why them and not others? And you're absolutely correct. They are kind of arbitrary.

The thing is, mathematician's also talk about weaker versions of topological spaces (with few axioms) and stronger ones (with additional axioms). For example, a pre-topological space is a topological space without the requirement that $\cl(\cl(S)) = \cl(S)$. On the other hand a $T1$ topological space is one which has the additional requirement that $\cl(\{x\}) = \{x\}$, that no other points are `infinitely close' to an individual point. One is free to study any of these that they like. It just that topological spaces seem to be a healthy compromise to start at.

\section{Topological Anatomy: Interior, Boundary...}

Up to this point, we've only talked about closure. You may be getting tired of it. Never fear, we're about to greatly expand the topological operations we have available.

To begin with, let us define the \emph{boundary} of a set as the points that adhere to both the set and its complement. $$\bd(S) = \cl(S)∩\cl(S^C)$$

We will also define the \emph{interior} operator, which will be the opposite of our closure operator. Instead of adding the boundary to a set, it removes the boundary from a set. We can find the interior of a set by considering the closure of its complement. By adding the boundary to the complement of the set, when we complement again we get our set without the boundary. Alternatively, we can define the interior as the set minus the boundary. $$\intr(S) = \cl(S^C)^C = S \setminus \bd(S)$$

Finally, the limit points of a set are the parts of the closure that are not in the original set. $$\lim(S) = \cl(S) \setminus S$$

To build some intuition, let's consider an example: the space is a rectangle, with our set as the bottom half, its complement as the top. They both contain part of the small stripe boundary.

...
%   __________________  
%  |            A'    |
%  |---------         |
%  |        |         |
%  |        ----------|
%  |    A             |
%  ^^^^^^^^^^^^^^^^^^^^

Let's now consider what happens to the set when we apply a number of operations.

...
%  __________________    __________________    __________________
% |    int(A)        |  |    int(A)        |  |                  |
% |------------------|  |------------------|  |       cl(A)      |
% | lim(A) | lim(A') |  |  bd(A) = bd(A')  |  |                  |
% |------------------|  |------------------|  |------------------|
% |    int(A')       |  |    int(A')       |  |    int(A')       |
% ^^^^^^^^^^^^^^^^^^^^  ^^^^^^^^^^^^^^^^^^^^  ^^^^^^^^^^^^^^^^^^^^

{\bf Exercise}: How many different sets can you make using $\cl$ and the standard set operations? %smaller sets hint?

{\bf Exercise}: A topological space can be defined in terms of an interior operator as well as closure. What are the axioms for an interior-based definition? (eg. $\intr(\intr(S)) = \intr(S)$ )

{\bf Exercise}: A topological space can also be defined in terms of a boundary operator as well as closure. What are the axioms for a boundary-based definition? (Hint: one of the axioms becomes unecessary.)


\section{Open and Closed}

An open set is one which is its own interior ($\intr(S) = S$). This means that it does not contain its boundary. Conversely, a closed set is one which is its own closure ($\cl(S) = S$). Closed sets contain their boundary.

A clopen set is a set that is both open and closed. How can this be? How can a set both contain and not contain its boundary? The answer is that clopen sets \emph{don't have a boundary}. For example, the empty set $∅$ and the full space $X$ are both clopen sets, since they have no boundary. The existence of other clopen sets is tightly tied to notions of connectedness, since if some portion of a space does not have a boundary, it is not connected to the rest of the space.

It is important to note that not all sets are open or closed (or both). Many contain part of their boundary but not another part.

{\bf Exercise}: Are the following subsets of the real numbers (with the usual topology) open, closed, clopen or neither?
\begin{enumerate}
\item $[0,1] = \{x | 0 ≤ x ≤ 1\}$
\item $[0,1) = \{x | 0 ≤ x < 1\}$
\item $(0,1) = \{x | 0 < x < 1\}$.
\end{enumerate}

{\bf Exercise}: We've considered sets which are their own interior (open sets) and sets which are their own closure (closed sets), but what about sets that are their own boundary (boundary sets)? Are they open, closed, clopen, or neither (and how does this vary between examples)?



\section{What Does Closure Look Like?}

\chapter{Finding Openess}

\section{Arbitrarily Close}

\section{Topology in Terms of Open Sets}




\end{document}
